% Created 2014-04-04 Fri 19:09
\documentclass[11pt,twocolumn]{article}
\usepackage[utf8]{inputenc}
\usepackage[T1]{fontenc}
\usepackage{fixltx2e}
\usepackage{graphicx}
\usepackage{longtable}
\usepackage{float}
\usepackage{wrapfig}
\usepackage{soul}
\usepackage{textcomp}
\usepackage{marvosym}
\usepackage{wasysym}
\usepackage{latexsym}
\usepackage{amssymb}
\usepackage{hyperref}
\usepackage{tikz}
\usepackage{color}
\usepackage{listings}


\graphicspath{ {images/} }

\tolerance=1000
\providecommand{\alert}[1]{\textbf{#1}}

\title{Packet Filtering Module}
\author{Aakarsh Nair}
\date{\today}

\begin{document}

\lstdefinestyle{customc}{
%  basicstyle=\footnotesize,       % the size of the fonts that are used for the code
  numbers=right,                   % where to put the line-numbers
  numberstyle=\footnotesize,      % the size of the fonts that are used for the line-numbers
  stepnumber=1,                   % the step between two line-numbers. If it is 1 each line will be numbered
  numbersep=5pt,                  % how far the line-numbers are from the code
  backgroundcolor=\color{white},  % choose the background color. You must add \usepackage{color}
  showspaces=false,               % show spaces adding particular underscores
  showstringspaces=false,         % underline spaces within strings
  showtabs=false,                 % show tabs within strings adding particular underscores
  frame=single,           % adds a frame around the code
  tabsize=2,          % sets default tabsize to 2 spaces
  captionpos=b,           % sets the caption-position to bottom
  breaklines=true,        % sets automatic line breaking
  breakatwhitespace=false,    % sets if automatic breaks should only happen at whitespace
  escapeinside={\%*}{*)},          % if you want to add a comment within your code
  %% belowcaptionskip=1\baselineskip,
  %% breaklines=true,
  %% frame=L,
  %% xleftmargin=\parindent,
  language=C,
  %% showstringspaces=false,
  basicstyle=\footnotesize\ttfamily,
  keywordstyle=\bfseries\color{green!40!black},
  commentstyle=\itshape\color{purple!40!black},
  %% identifierstyle=\color{black},
  stringstyle=\color{orange},
}

\lstdefinelanguage{anX86} {
	morekeywords={
		nop,movl,pushl,popl,cmpl,testl,leal,cmovl,jmp,je,jne,
		jz,jnz,jg,jge,jl,jle,addl,subl,imul,idiv,cdq,incl,decl,
		negl,andl,orl,xorl,notl,shrl,shll,sarl,sall,ret,leave,
		call,setg,setl,setge,setle,setne,sete,movzbl,
		eax,ebx,ecx,edx,edi,esi,ebp,esp,al,bl,cl,dl,
                pause,lock,decb,jns,jle,jmp,cmpb,movb,xchg,xchgb,lea
	},
	sensitive=true,
	morecomment=[l]{\#},
	morestring=[b]"
}
\lstset{escapechar=@,style=customc}
\lstset{language=C}

\maketitle

\setcounter{tocdepth}{3}
\tableofcontents

\maketitle
\vspace*{1cm}

\section{Introduction}

The \lstinline{pk_filter} module is a test case for supporting the
addtion of custom packet filtering to the linux kernel. The module is
meant to be configurable from user space. The module is configured and
operated over netlink sockets. Most of the module was developed as
part of the UCSC Linux Kernel class.


\section{Historical}

\subsection{Berekeley Packet Filter}



\section{Starting and Starting}

\section{Rule Description}

\section{User Space Filters}

\subsection{Summary and Acknowledgment}

Most of the module

\begin{thebibliography}{9}

\bibitem{libnl}
  \textit{LibNL : Userspace netlink library documentation} \\
  \url{http://www.infradead.org/~tgr/libnl/doc/core.html} 


\bibitem{bpf}
  \textit{The BSD Packet Filter: A New Architecture for User-level Packet Capture} \\
  \textit{Steven McCanne and Van Jacobson} \\
  \url{http://www.tcpdump.org/papers/bpf-usenix93.pdf}

  
\end{thebibliography}

\end{document}
